\documentclass[a4j]{jarticle}

\usepackage[dvipdfmx]{graphicx}
\usepackage{epsbox}
\usepackage{url}
\usepackage{float}
\usepackage{here}
\usepackage{ascmac}
\setlength{\headsep}{-5mm}
\setlength{\oddsidemargin}{0mm}
\setlength{\textwidth}{165mm}
\setlength{\textheight}{230mm}
\setlength{\footskip}{20mm}
\title{
  \vspace{30mm}
         {\bf システム提案書}
         \\
}
\author{
  \vspace{5mm}
  007 \\
  \vspace{5mm}
  学籍番号 1200381 \\
  \vspace{5mm}
         {\large 横田 励矢}
         \vspace{10mm}
}
\date{
}

\begin{document}
\maketitle

\newpage

\section{はじめに}
近年、女性の社会進出などの労働環境の変化に伴い、夫婦の共働きが増加しています。それにより、保育園の待機児童が増加し保護者の育児負担が大きくなっています。また、こういった社会の変化に伴い、保護者同士のコミュニケーションが減少しています。このような保護者の負担の軽減、保護者同士のコミュニケーション、子供の発育を目的とした子育て支援システムをご提案します。

\newpage
\section{利用者}
\begin{itemize}
\item 2~5歳の幼児\\
  保護者の家事をする時間などに知育ゲームを利用
\item 2~5歳の保護者\\
  SNSを用いた保護者同士のつながりに利用
\end{itemize}


\newpage
\section{このシステム提案のアピールポイント}
\begin{description}
\item[(1)]保護者の家事などの合間に子供たちに利用してもらうため保護者の負担を軽減できることが考えられます。
\item[(2)]子供の発育を促すだけでなく、子供のゲームの記録を残すことができるため子供の日々の成長過程をいつでも見ることが可能となります。
\item[(3)]SNS機能を設けているため、保護者同士のコミュニケーションがとりやすくなることが考えられます。さらに質問箱などの機能も設けているため、子供が熱を出したときの対処等を聞くなど保護者同士でのサポートが可能となります。
\end{description}

\end{document}
