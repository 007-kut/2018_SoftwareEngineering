\documentclass[a4j,titlepage]{jarticle}
%% プリアンブルここから
\usepackage[dvipdfmx]{graphicx}
\usepackage{url}
\usepackage{ascmac}


%% 本文
\begin{document}
\section{8 費用と効果}
アプリの開発及び運用にかかる費用は以下の通りです。
 \begin{table}[htp]
\begin{center}
  \caption{STPケーブルの順番}
  \begin{tabular}{|l|r|l|r|l|}\hline
    項目& 単価(円) & 数量 & 金額(円) & 備考  \\ \hline 
    Android端末(本体)& 30,000 & 3台 & 90,000 &   \\ \hline
    Android端末(通信料)& 100,000 & 1台分 & 100,000 & 1台のみインターネット通信を試行するため  \\ \hline
    システム開発人件費& 40,000 & 420人日 & 16,800,000 & 工程内訳:7人×2ヶ月(60日)  \\ \hline
    サーバー代& 250,000 & 1台 & 250,000 &   \\ \hline
    維持費& 1,000,000 & 5年 & 5,000,000 &     \\ \hline
    \multicolumn{3}{|l|}{合計} & 22,240,000 &  \\ \hline
  \end{tabular}
\end{center}
\end{table}

このアプリを運用することにより待機自動の教育不足や親同士でのつながりの現象を解決することできると想定さる。このアプリは300円の有料アプリで全国の2~5才児の子どもを持つ保護者を対象としており、その内の3\%となる約12万人の方がダウンロードすると仮定し、Googleで開発すると3割取られるため算出結果は以下のようになります。

\begin{equation}
  120,000×300×0.7 = 25,200,000
\end{equation}

開発と運用にかかる費用と照らし合わせると2,960,000円の黒字となり利益を出すことができます。
また、この結果は1年間運用した場合の結果であり,出生率を見ると毎年約90万人近くの子供が産まれているため5年間運用した場合

\begin{equation}
  228,000×300×0.7 = 47,880,000
\end{equation}

となるためさらなる利益が期待できると考えられます。


\end{document}



