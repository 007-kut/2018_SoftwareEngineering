\documentclass[a4j]{jarticle}

\usepackage[dvipdfmx]{graphicx}
\usepackage{epsbox}
\usepackage{url}
\usepackage{here}

\setlength{\headsep}{-5mm}
\setlength{\oddsidemargin}{0mm}
\setlength{\textwidth}{165mm}
\setlength{\textheight}{230mm}
\setlength{\footskip}{20mm}

\title{
\vspace{30mm}
{\bf データテーブル}
\date{}
}

\begin{document}
\maketitle
\section{データテーブル構造}

以下に本システムで用いるデータテーブルを示す。\\
表\ref{tbl: user}はユーザー情報を示しており、親の情報を格納している。\\

\begin{table}[H]
    \caption{親ユーザー情報}
    \label{tbl: user}
    \begin{center}
        \begin{tabular}{|c|c|c|c|c|} \hline
            属性 & データ型 & データ長 & SQLite & key:Table名\\ \hline \hline
            UserID & 半角英数字型 & 6文字固定長 & TEXT & PK\\ \hline
            Name & 全角文字型 & 10文字可変長 & TEXT & \\ \hline
            Area & 全角文字型 & 5文字可変長 & TEXT & \\ \hline
        \end{tabular}
    \end{center}
\end{table}


表\ref{tbl: question}~表\ref{tbl: date}では、質問箱に関するデータを示している。表\ref{tbl: question}では質問と質問者の関連付けを、表\ref{tbl: answer}では回答と回答者の関連付けを行い、質問と回答の管理を行っている。\\
また、表\ref{tbl: qcontents}では質問内容を、表\ref{tbl: acontents}では回答内容を、表\ref{tbl: date}では日付の管理を行っている。\\
\begin{table}[H]
    \caption{質問}
    \label{tbl: question}
    \begin{center}
        \begin{tabular}{|c|c|c|c|c|} \hline
            属性 & データ型 & データ長 & SQLite & key:Table名\\ \hline \hline
            QID & 半角英数字型 & 8文字固定長 & TEXT & PK\\ \hline
            UserID & 半角英数字型 & 6文字固定長 & TEXT & FK:親ユーザー情報\\ \hline
            QcontentsID & 半角英数字型 & 10文字固定長 & TEXT & FK:質問内容\\ \hline
            DateID & 半角英数字型 & 8文字固定長 & TEXT & FK:日付\\ \hline
        \end{tabular}
    \end{center}
\end{table}

\begin{table}[H]
    \caption{回答}
    \label{tbl: answer}
    \begin{center}
        \begin{tabular}{|c|c|c|c|c|} \hline
            属性 & データ型 & データ長 & SQLite & key:Table名\\ \hline \hline
            AID & 半角英数字型 & 8文字固定長 & TEXT & PK\\ \hline
            QID & 半角英数字型 & 8文字固定長 & TEXT & FK:質問\\ \hline
            UserID & 半角英数字型 & 6文字固定長 & TEXT & FK:親ユーザー情報\\ \hline
            AcontentsID & 半角英数字型 & 10文字固定長 & TEXT & FK:回答内容\\ \hline
            DateID & 半角英数字型 & 8文字固定長 & TEXT & FK:日付\\ \hline
        \end{tabular}
    \end{center}
\end{table}

\begin{table}[H]
    \caption{質問内容}
    \label{tbl: qcontents}
    \begin{center}
        \begin{tabular}{|c|c|c|c|c|} \hline
           属性 & データ型 & データ長 & SQLite & key:Table名\\ \hline \hline
            QcontentsID & 半角英数字型 & 10文字固定長 & TEXT & PK\\ \hline
            Qcontents & 全角文字型 & 100文字可変長 & TEXT & \\ \hline
        \end{tabular}
    \end{center}
\end{table}


\begin{table}[H]
    \caption{回答内容}
    \label{tbl: acontents}
    \begin{center}
        \begin{tabular}{|c|c|c|c|c|} \hline
            属性 & データ型 & データ長 & SQLite & key:Table名\\ \hline \hline
            AcontentsID & 半角英数字型 & 10文字固定長 & TEXT & PK\\ \hline
            Acontents & 全角文字型 & 100文字可変長 & TEXT & \\ \hline
        \end{tabular}
    \end{center}
\end{table}

\begin{table}[H]
    \caption{日付}
    \label{tbl: date}
    \begin{center}
        \begin{tabular}{|c|c|c|c|c|} \hline
            属性 & データ型 & データ長 & SQLite & key:Table名\\ \hline \hline
            DateID & 半角英数字型 & 8文字固定長 & TEXT & PK\\ \hline
            Date & 日付型 & 20文字固定長 & NUMERIC & \\ \hline
        \end{tabular}
    \end{center}
\end{table}


また、表\ref{tbl: datatable}に上記のデータテーブルに対する操作を示す。

\begin{table}[H]
    \caption{表14 各種データテーブルに対する操作}
    \label{tbl: datatable}
    \begin{center}
        \begin{tabular}{|l||c|c|c|c||c|c|c|c|} \hline
             & \multicolumn{4}{|c||}{アプリケーション} & \multicolumn{4}{|c|}{管理画面}\\ \hline
            データテーブル & \multicolumn{1}{|l|}{作成} & \multicolumn{1}{|l|}{参照} & \multicolumn{1}{|l|}{更新} & \multicolumn{1}{|l||}{削除} & \multicolumn{1}{|l|}{作成} & \multicolumn{1}{|l|}{参照} & \multicolumn{1}{|l|}{更新} & \multicolumn{1}{|l|}{削除}\\ \hline \hline
            親ユーザー情報 & 〇 & 〇 & 〇 & 〇 & 〇 & 〇 & 〇 & 〇\\ \hline
            質問 & 〇 & 〇 & 〇 &  & 〇 & 〇 & 〇 & 〇\\ \hline
            質問内容 & 〇 & 〇 & 〇 &  & 〇 & 〇 & 〇 & 〇\\ \hline
            回答 & 〇 & 〇 & 〇 &  & 〇 & 〇 & 〇 & 〇\\ \hline
            回答内容 & 〇 & 〇 & 〇 &  & 〇 & 〇 & 〇 & 〇\\ \hline
            日付 &  & 〇 & 〇 &  & 〇 & 〇 & 〇 & 〇\\ \hline
        \end{tabular}
    \end{center}
\end{table}

\end{document}
