\documentclass[a4j]{jarticle} 
\usepackage[dvipdfmx]{graphicx} %必要 
\usepackage[dvipdfmx]{color} %必要 
\usepackage{url} %実験のテンプレに記載 

\setlength{\headsep}{-5mm} 
\setlength{\oddsidemargin}{0mm} 
\setlength{\textwidth}{165mm} 
\setlength{\textheight}{230mm} 
\setlength{\footskip}{20mm} 


\begin{document} 
\section{システム概要} 
本システムは機能としてゲーム機能、成長記録機能、SNS(子育て窓口)機能、設定機能、問い合わせ機能の5つを実装します。この章では、各機能の説明及び各機能を構成するサブシステムを示します。サブシステムの詳細は第〜章にて説明します。 

\subsection{成長記録機能} 
成長記録機能では、日々の子供の成長を記録できる機能を提供します。成長記録は日付ごとに分けて記録され、いつでも見返すことができます。他のSNSに記録を共有することもできます。 
\subsubsection*{構成サブシステム} 
\noindent カメラ撮影サブシステム、コメント記録サブシステム、ゲーム結果所得サブシステム、カレンダーサブシステム、共有サブシステム 

\subsection{ゲーム機能}
ゲーム機能では、2〜5歳を対象とした簡単な知育ゲームを提供します。お絵描きやおつかい等のゲームがあり、お絵描きではゲームの記録を成長記録機能に保存することが可能です。
\subsubsection*{構築サブシステム}
\noindent ペイントサブシステム、おつかいサブシステム ゲーム記録保存サブシステム

\subsection{SNS機能}
SNS機能では、コミュニティ(アカウントに登録した住んでいる地域)に質問を投稿・回答する質問箱を提供します。質問の検索も可能です。
\subsubsection*{構築サブシステム}
\noindent 検索サブシステム 昇順サブシステム 投稿サブシステム 回答サブシステム

\subsection{設定機能} 
本システムの機能の1つであるSNS機能を利用するためには、アカウント登録を行う必要があります。設定機能では、アカウントの登録、変更、削除を行うことができます。本システムを初めて利用する場合は、設定機能において初期設定を行うところから始まります。 
\subsubsection*{構成サブシステム} 
\noindent アカウント登録サブシステム、登録情報変更サブシステム

\subsection{問い合わせ機能} 
本システムの利用中に不具合が生じた場合やシステムの使い方に疑問等がある場合は、問い合わせ機能を利用し管理者に報告することができます。問い合わせ情報を基に、早急に管理者が対応します。 
\subsubsection*{構成サブシステム} 
\noindent 外部メール連携サブシステム 


\section{サブシステム概要} 
第〜章にて述べた5つの機能それぞれを構成しているサブシステムについて説明します。 

\subsection{成長記録機能} 
\subsubsection*{カメラ撮影サブシステム} 
利用端末に搭載されているカメラをシステム内から起動し、写真を撮影することができます。 
\subsubsection*{コメント記録サブシステム} 
保存されている成長記録(写真、ゲームの結果)に対してコメントを記述することができます。成長記録を見ながらコメントを読み返すことにより、過去の思い出が振り返りやすくなります。 
\subsubsection*{ゲーム結果所得サブシステム} 
ゲーム機能で子供が遊んだ結果を成長記録として保存することができます。 
\subsubsection*{カレンダーサブシステム} 
成長記録を日付ごとに分けるサブシステムです。本システムの利用者は、画面に表示される日付を選択することによって、各日に保存された成長記録を確認することができます。 
\subsubsection*{共有サブシステム} 
成長記録を他のSNSアプリケーションで共有することができます。しかし、個人が特定されるような写真の共有は推奨しません。 

\subsection{ゲーム機能} 
\subsubsection*{ペイントサブシステム} 
お絵描きのゲームを提供します。お手本を見ながら同じものを描く機能と自由に描く機能を設けます。
\subsubsection*{おつかいサブシステム}
おつかいのゲームを提供します。特定の料理の材料を買うことで料理が完成するという内容です。
\subsubsection*{ゲーム記録保存サブシステム}
お絵描きの記録を保存するサブシステムです。保存されることにより成長記録機能での記録の閲覧が可能です。

\subsection{質問箱機能} 
\subsubsection*{検索サブシステム}
質問の検索を行うサブシステムです。 質問したい内容・ワードを入力することで質問の内容・ワードに合致したものが表示されます。
\subsubsection*{昇順サブシステム}
投稿された質問を昇順に並び替えるサブシステムです。
\subsubsection*{投稿サブシステム}
質問をコミュニティに投稿するサブシステムです。質問はコミュニティ内でのみ表示されます。
\subsubsection*{回答サブシステム}
質問に回答を行うサブシステムです。質問の回答はコミュニティ内でのみ可能です。

\subsection{設定機能} 
\subsubsection*{アカウント登録サブシステム} 
システムを初めて利用する際にアカウント登録を行います。登録する情報は、名前(ニックネーム可)と住んでいる都道府県です。 
\subsubsection*{登録情報サブシステム} 
登録情報の変更及びアカウントの削除を行うことができます。しかし、SNS機能で投稿した内容などは削除されません。 

\subsection{問い合わせ機能} 
\subsubsection*{外部メール連携サブシステム} 
管理者に対する問い合わせは、利用端末に搭載されているメール機能を利用して行います。問い合わせアイコンを選択すると自動的にメール機能が起動します。メール内に問い合わせ内容を記述し送信することによって、管理者に報告することができます。 










\end{document} 
