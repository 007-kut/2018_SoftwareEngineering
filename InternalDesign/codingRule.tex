
\documentclass[a4j]{jarticle}

\usepackage[dvipdfmx]{graphicx}
\usepackage{epsbox}
\usepackage{url}
\usepackage{here}

\setlength{\headsep}{-15mm}
\setlength{\oddsidemargin}{0mm}
\setlength{\textwidth}{165mm}
\setlength{\textheight}{250mm}
\setlength{\footskip}{20mm}

\title{
\vspace{5mm}
{\bf おやこね!コーディング規約}
\date{}
}

\begin{document}
\maketitle
\section{コーディング規約}
本章では、本システムを開発する際のコーディング規約を示します。

\subsection{命名規則}
\begin{itemize}
  %スネーク記法のほうが良ければスネーク記法にしましょ(例:setting_account)
  \item クラス名・モジュール名
  %アッパーキャメルケース表記で命名を行います。
  \begin{itemize}
    \item 2つ以上の英単語を使用します。
    \item 単語の頭文字は全て大文字を使用します。
    \item 名称は英字のみを使用し、意味のあるものを用います。
  \end{itemize}
  (例:SettingAccount)

  \item メソッド名・変数名
  %ローワーキャメルケース表記で命名を行います。
  \begin{itemize}
    \item 2つ以上の英単語を使用します。
    \item 名称の頭文字は小文字、後続する単語の頭文字は大文字を使用します。
    \item 名称は英字のみを使用し、意味のあるものを用います。
  \end{itemize}
  (例:getStatus)

  \item 定数
  \begin{itemize}
    \item 全ての文字を大文字で表記します。
    \item 2つ以上の単語を用いる場合は、単語と単語を'\_'で区切ります。
  \end{itemize}
  (例:ERRAND\_ID)
  %errand -> おつかいの意味
\end{itemize}

\subsection{コーディングスタイル}

\begin{itemize}
  \item インデント
  \begin{itemize}
    \item インデントはTAB文字、または半角スペース4文字を使用します。
  \end{itemize}

  \item 括弧
  \begin{itemize}
    \item 中括弧は改行して開始します。
    \item 小括弧の前後にはスペースを使用しない。
  \end{itemize}

  \item 演算子
  \begin{itemize}
    \item 演算子の前後には半角スペースを一文字分使用します。
  \end{itemize}

  %\item 文字コード
\end{itemize}


\end{document}
